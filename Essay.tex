% Please do not change the document class
\documentclass{scrartcl}

% Please do not change these packages
\usepackage[hidelinks]{hyperref}
\usepackage[none]{hyphenat}
\usepackage{setspace}
\doublespace

% You may add additional packages here
\usepackage{amsmath}

% Please include a clear, concise, and descriptive title
\title{What is the value of guidelines created for game developers portraying suicide and/or depression?}

% Please do not change the subtitle
\subtitle{COMP230 - Ethics and Professionalism}

% Please put your student number in the author field
\author{1706966}

\begin{document}
	
	\maketitle
	
	\abstract{}
	
	\section{Introduction}
		Depression is a common illness worldwide, affecting more than three hundred million people and close to eight hundred thousand people dying due to suicide every year\cite{DepressionStats}. All avenues that can help to diminish these issues should be explored. This paper looks at the potential value that could be held in video game industry specific guidelines in regards to depicting suicide and/or depression.
		
		In severe cases, depression can lead to suicide. In the U.S it is estimated that 60\% of people who commit suicide are suffering with severe depression\cite{SuicideToDepression}, this link gives strong reasoning to look at both suicide and depression for the guidelines. Suicide is the leading cause of death among young people aged 20-34 years in the UK\cite{youngAge}, with about 29\%\cite{sales2018essential} of the 2.2 billion\cite{globalGame} people playing video games falling around that age group. Therefore working to ensure video games are not causing a negative mental health effect could have a large impact on the number of people suffering.
	
	\section{Current Guidelines}
		Pan European Game Information (PEGI) do not have a label to warn of the depiction of suicide or depression in video games\cite{pegiNoRating}. The role of PEGI labels is to advise consumers of potentially distressing content in games. With a warning for depression/suicide being non-existent it shows how little this issue has been considered in the larger governing bodies.
		
		Currently no guidelines exist that are specific to the video game industry in regards to how suicide and depression are handled in games. Popularised guidelines are already available for referring to when portraying suicide in the news\cite{world2017preventing,nepon2009media} and drama adaptations\cite{DramaGuidelines}.		
		The drama adaptation guidelines match best with a video game environment as both generally show fictional scenes. However due to the artistic style, interactivity and immersion of video games it would be best to form guidelines more specific to the industry. 	

	\section{Impact of showing suicide/depression}
		A systematic review\cite{pirkis2001suicide} has shown evidence of the reporting and portrayal of suicidal behaviour to have a negative effect and facilitate suicide among viewers. Conversely video games have been proven to be able to cause positive health-related behaviour change\cite{baranowski2008playing}. These video games were designed specifically to facilitate positive behaviour changes. With this information in mind, it is logical to expect that when video games portray suicidal behaviour without consideration to its effects on the viewer, we could likely see similar negative effects to that stated in the aforementioned review\cite{pirkis2001suicide}.
		
	%doki doki

	\section{Will the industry consult new guidelines}
		Due to the nature of guidelines being optional, adherence to existing guidelines in the video games industry appears to be loose. If universal compliance is desired this will need to be legislation led. A prime example of this is the presence of game accessibility guidelines. These guidelines have existed for many years\cite{accessGuidelines1,accessGuidelines2}, however until this year when legislation was formed, there was a drive in research to push forward the issue of accessibility into law\cite{powers2015video}. With this in mind guidelines are a good place to start and some games did follow the guidelines, a great example of these can be shown by viewing the winners of annual AbleGamers awards\cite{ableGamers}. There is a strong argument that the availability of guidelines existing also helped to show how important of an issue accessibility in games was when fighting to have laws implemented. Even if not contributing to legislation guidelines can help raise awareness of issues throughout industry.
		
		A major factor that may sway the industry in how they receive new guidelines, is that games are often seen as an art form\cite{pearce2006games}. There is a worry that developers will consider the guidelines to be restrictive, making them unlikely to consult the guidelines. For guidelines to stand a chance of being accepted it is vastly important that they are given credibility. To provide credibility to the guidelines they will need to be based entirely on evidence, additionally they should be created with collaboration between researchers, public health policy makers and game development industry experts\cite{hawton2002influences}. One way to enforce guideline credibility is to prove the protective value that guidelines have on consumer health. However this is the most difficult to achieve.
		
	\section{Proving guideline value}
		To bring to life how guidelines when used well can cause a positive effect on players, it is best to look for research papers that have shown a difference between before and after the implementation of guidance.
		An example is how when researchers and journalists have worked to limit the reporting of subway suicides, resulting in a reduction of suicides using that method\cite{etzersdorfer1998preventing}. This displays a correlation between the representation of suicide methods in media and the use of these particular methods in individuals.
		Similarly researchers recorded a drop in suicide rates following the introduction of Australian media guidelines\cite{niederkrotenthaler2007assessing}, helping to solidify the argument that well made guidelines have a positive effect on players health. 
		
	\section{Conclusion}
		The discussion of this paper has shown that guidelines could potentially have a place in helping to steer the creation of content to have a protective influence on end users. The fact that no guidelines exist for video game developers in regards to portraying suicide and/or depression is a great oversight, and certainly goes to show how under represented this issue is. 
		Guidelines created to fill this gap stand a good chance of not being taken on board by many developers, but will help negate negative effects when they are referred to, assuming they are steeped in credibility. Having guidelines available will help to cultivate discussion and actions centred within this topic, raising awareness of the issue.
		
		
	\bibliographystyle{ieeetran}
	\bibliography{references}
	
\end{document}
