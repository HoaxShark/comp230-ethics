% Please do not change the document class
\documentclass{scrartcl}

% Please do not change these packages
\usepackage[hidelinks]{hyperref}
\usepackage[none]{hyphenat}
\usepackage{setspace}
\doublespace

% You may add additional packages here
\usepackage{amsmath}

% Please include a clear, concise, and descriptive title
\title{Should guidelines be created for video game developers to
	refer to when portraying suicide and/or depression?}

% Please do not change the subtitle
\subtitle{COMP230 - Ethics and Professionalism}

% Please put your student number in the author field
\author{1706966}

\begin{document}
	
	\maketitle
	
	\abstract{}
	
	\section{Introduction}
		With depression being a common illness worldwide, affecting more than three hundred million people along with close to eight hundred thousand people dying due to suicide every year\cite{DepressionStats}, all avenues that can help to diminish this problem should be explored. This paper takes a look into what possible effects could be seen if guidelines were created for the video game industry to consider when portraying suicide and/or depression in video games.
		
		In severe cases, depression can lead to suicide. In the U.S it is estimated that 60\% of people who commit suicide are suffering with severe depression\cite{SuicideToDepression}, this link gives strong reasoning to look at both suicide and depression for the guidelines. Suicide is the leading cause of death among young people aged 20-34 years in the UK\cite{youngAge}, with about 29\%\cite{sales2018essential} of the 2.2 billion\cite{globalGame} people playing video games falling around that age group. Therefore working to ensure video games are not causing a negative mental health effect could have a large impact on the number of people suffering.
	
	\section{Current Guidelines}
		Pan European Game Information (PEGI) don't have a label to warn of the depiction of suicide or depression in video games\cite{pegiNoRating}. The reason PEGI labels exist is to warn parents and players of what is in the game. With a warning for depression/suicide being non-existent it shows how little this issue has been considered in the larger governing bodies.
		
		Currently no guidelines exist that are specific to the video game industry in regards to how suicide and depression are handled in games. Popularised guidelines are already available for referring to when portraying suicide in the news\cite{world2017preventing}\cite{nepon2009media} and drama adaptations\cite{DramaGuidelines}.		
		The drama adaptation guidelines match best with a video game environment as both generally show fictional scenes. However due to the artistic style, interactivity and immersion of video games it would be best to form guidelines more specific to the industry. 	

	\section{Impact of showing suicide/depression}
		A systematic review\cite{pirkis2001suicide} has shown evidence of the reporting and portrayal of suicidal behaviour to have a negative effect and facilitate suicide among viewers. Conversely video games have been proven to be able to cause positive health-related behaviour change\cite{baranowski2008playing}, these video games were designed specifically to facilitate positive behaviour changes. With this information in mind, it is logical to expect that when video games portray suicidal behaviour without consideration to its effects on the viewer, we could likely see similar negative effects to that stated in the aforementioned review\cite{pirkis2001suicide}.
		
	\section{Controversy}
	%games to look at:
	%doki doki
	%life is strange
	%dishonored?
	%beyond 2 souls
	\section{Will the industry consult new guidelines}
		The video games industry has been known to ignore guidelines until the guidances become law. A prime example of this is the presence of game accessibility  guidelines. With a large amount of guidelines existing for years now\cite{accessGuidelines1}\cite{accessGuidelines2} there was still a drive in research to push forward the issue of accessibility to be brought into law\cite{powers2015video}. Given the fact that laws forcing the industry to work on the accessibility of their games still needed to be created, proves that without enforcement through legal manners the change throughout the industry is likely to be lacklustre. However beginning by creating guidelines has shown to be a good start, some games did follow the guidelines, a great example of these can be shown by viewing the winners of annual AbleGamers awards\cite{ableGamers}. There is a strong argument that the availability of guidelines existing also helped to show how important of an issue accessibility in games was when fighting to have laws implemented.
		
		A major factor that may sway the industry in how they receive new guidelines, is that games are often seen as an art form\cite{pearce2006games}. There is a worry that developers will consider the guidelines to be restrictive, making them unlikely to consult the guidelines. For guidelines to stand a chance of being accepted it is vastly important that they are given credibility. To provide credibility to the guidelines they will need to be based entirely on evidence, additionally they should be created with collaboration between researchers, public health policy makers and game development industry experts\cite{hawton2002influences}. Lastly and arguably the trickiest way to  provide credibility to the guidelines is to prove that they work.
		
	\section{Proving guidelines work}
		To bring to life how guidelines, when used well, can cause a positive effect on society, it is best to look for research papers that have shown a difference between before and after the implementation of guidance.
		Collaboration between researchers and journalists limiting reporting of subway suicides was followed by a reduction in the number of suicides using that method\cite{etzersdorfer1998preventing}, showing evidence that withholding details about specific suicide methods at least lowers the amount that method is used. 
		Researchers recorded a drop in suicide rates following the introduction of Australian media guidelines\cite{niederkrotenthaler2007assessing}, helping to solidify the argument that well made guidelines have a positive effect on the public. 
		
	\section{Conclusion}
		It is obvious from the discussion of this paper that guidelines can have a place in helping to steer the creation of content to a more positive impact on end users. The fact that no guidelines exist for video game developers in regards to portraying suicide and/or depression is a great oversight, and certainly goes to show how under represented this issue is. 
		Guidelines created to fill this gap stand a good chance of not being taken on board by many developers, but will have a positive effect when they are referred to, assuming they are steeped in credibility. Once guidelines have been available for a while further research can be done to consider if the industry has seen any changes in how they represent suicide/depression and any effects this has caused, to either the end user or the industry itself. Hopefully driving more research for this topic. 
		
	\bibliographystyle{ieeetran}
	\bibliography{references}
	
\end{document}
