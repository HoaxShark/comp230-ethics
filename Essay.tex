% Please do not change the document class
\documentclass{scrartcl}

% Please do not change these packages
\usepackage[hidelinks]{hyperref}
\usepackage[none]{hyphenat}
\usepackage{setspace}
\doublespace

% You may add additional packages here
\usepackage{amsmath}

% Please include a clear, concise, and descriptive title
\title{heh}

% Please do not change the subtitle
\subtitle{COMP230 - Ethics and Professionalism}

% Please put your student number in the author field
\author{1706966}

\begin{document}
	
https://www.ncbi.nlm.nih.gov/pmc/articles/PMC1124845/	
https://www.mentalhealth.org.uk/a-to-z/s/suicide
http://www.theesa.com/wp-content/uploads/2018/05/EF2018_FINAL.pdf
	
	\maketitle
	
	\abstract{}
	
	\section{Introduction}
		With depression being a common illness worldwide, affecting more than three hundred million people and close to eight hundred thousand people dying due to suicide every year\cite{DepressionStats}, all avenues that can help to diminish this problem should be explored. This paper takes a look into what possible effects could be seen if guidelines were created for the video game industry to consider when portraying suicide and/or depression in video games.
		
		In severe cases, depression can lead to suicide. In the U.S it is estimated that 60\% of people that commit suicide are suffering with severe depression\cite{SuicideToDepression}, this link gives additional reasoning to look at both suicide and depression for the guidelines. Suicide is the leading cause of death among young people aged 20-34 years in the UK\cite{youngAge}, with about 29\%\cite{sales2018essential} of the 2.2 billion\cite{globalGame} people playing video games falling around that age group making sure video games are not causing a negative effect is incredibly important.
	
	\section{Current Guidelines}
	Pan European Game Information (PEGI) don't have a label to warn of the depiction of suicide or depression in video games\cite{pegiNoRating}. The reason the PEGI labels exist is to warn parents and players of what is in the game. With a warning for depression/suicide being missed it shows how little this issue has been considered.\vspace{5mm} %5mm vertical space
	
	Currently there are no guidelines specific to the video game industry in regards to how suicide and depression are handled in video games. Popularised guidelines do exist for portraying suicide in the news\cite{world2017preventing}\cite{nepon2009media} and drama adaptations\cite{DramaGuidelines}. \vspace{5mm} %5mm vertical space
	
	The drama adaptation guidelines match best with a video game environment as both generally show fictional scenes. However due to the artistic style, interactivity and immersion of video games it would be best to form guidelines more specific to the industry. 	

	games to look at:
	doki doki
	life is strange
	dishonored?
	beyond 2 souls
	\section{}
		A systematic review\cite{pirkis2001suicide} has shown evidence of the reporting and portrayal of suicidal behaviour to have a negative effect and facilitate suicide among viewers. Conversely video games have been proven to be able to cause positive health-related behaviour change \cite{baranowski2008playing}, these video games were designed specifically to facilitate positive behaviour changes.
	\section{}
	
	\section{Conclusion}

	\bibliographystyle{ieeetran}
	\bibliography{references}
	
\end{document}
